\documentclass[compress]{beamer}
\usepackage{ifthen,verbatim,ulem}

\newcommand{\isnote}{}
\xdefinecolor{lightyellow}{rgb}{1.,1.,0.25}
\xdefinecolor{darkblue}{rgb}{0.1,0.1,0.7}

%% Uncomment this to get annotations
%% \def\notes{\addtocounter{page}{-1}
%%            \renewcommand{\isnote}{*}
%% 	   \beamertemplateshadingbackground{lightyellow}{white}
%%            \begin{frame}
%%            \frametitle{Notes for the previous page (page \insertpagenumber)}
%%            \itemize}
%% \def\endnotes{\enditemize
%% 	      \end{frame}
%%               \beamertemplateshadingbackground{white}{white}
%%               \renewcommand{\isnote}{}}

%% Uncomment this to not get annotations
\def\notes{\comment}
\def\endnotes{\endcomment}

\setbeamertemplate{navigation symbols}{}
\setbeamertemplate{headline}{\mbox{ } \hfill
\begin{minipage}{5.5 cm}
\vspace{-0.75 cm} \small
\end{minipage} \hfill
\begin{minipage}{4.5 cm}
\vspace{-0.75 cm} \small
\begin{flushright}
\ifthenelse{\equal{\insertpagenumber}{1}}{}{Jim Pivarski \hspace{0.2 cm} \insertpagenumber\isnote/\pageref{numpages}}
\end{flushright}
\end{minipage}\mbox{\hspace{0.2 cm}}\includegraphics[height=1 cm]{../cmslogo} \hspace{0.1 cm} \includegraphics[height=1 cm]{../tamulogo} \hspace{0.01 cm} \vspace{-1.05 cm}}

\begin{document}
\begin{frame}
\vfill
\begin{center}
\textcolor{darkblue}{\Large Alignment Combination Plans with Contingencies}

\vfill
\begin{columns}
\column{0.3\linewidth}
\begin{center}
\large
\textcolor{darkblue}{Jim Pivarski}
\end{center}
\end{columns}

\begin{columns}
\column{0.3\linewidth}
\begin{center}
\scriptsize
{\it Texas A\&M University}
\end{center}
\end{columns}

\vfill
25 February, 2009

\end{center}
\end{frame}

%% \begin{notes}
%% \item This is the annotated version of my talk.
%% \item If you want the version that I am presenting, download the one
%% labeled ``slides'' on Indico (or just ignore these yellow pages).
%% \item The annotated version is provided for extra detail and a written
%% record of comments that I intend to make orally.
%% \item Yellow notes refer to the content on the {\it previous} page.
%% \item All other slides are identical for the two versions.
%% \end{notes}

\small

\begin{frame}
\frametitle{Outline}
\begin{itemize}\setlength{\itemsep}{0.75 cm}
\item Expanded version of the CSC alignment combination procedure I
  presented last CMS Week
\item Emphasis: each procedure must input and output \mbox{CSCAlignmentRcds,\hspace{-1 cm}}
  modifying only the parameters they best measure
\item Similar idea for DT alignment
\end{itemize}
%% \hspace{-0.83 cm} \textcolor{darkblue}{\Large Outline2}
\end{frame}

\begin{frame}
\frametitle{Available CSC Procedures}
{\scriptsize (all parameters are local: e.g.\ $x$ is $r\phi$, $y$ is radial)}

\renewcommand{\arraystretch}{1.25}

\begin{tabular}{c p{0.25\linewidth} p{0.2\linewidth} p{0.4\linewidth}}
$\surd$ & Chamber \mbox{Photogrammetry} & $x$, $y$, $z$, $\phi_x$, $\phi_z$ chambers relative to disks & \mbox{100\% valid at $\vec{B}=0$~T,} some parameters are invalid when $\vec{B}>0$ \mbox{(mostly $y$, $z$, $\phi_x$)\hspace{-1 cm}} \\\hline
& \mbox{Transfer lines} \mbox{and/or cavern} photogrammetry & 6 dof for disks & puts chamber coordinates into global frame, but need to connect physical measurements with (a)~chamber centers and (b)~the tracker \\\hline
$\surd$ & DCOPS/SLM disk bending & average $z$ and $\phi_x$ for rings & corrects for large deviations due to $\vec{B}$-field \\\hline
& StandAloneMuon cosmics alignment & 6 dof for disks or $x$, $z$, $\phi_y$, $\phi_z$ for chambers & barrel (reference) needs to be better aligned, track refitter may need more debugging \\\hline
$\surd$ & Beam-halo Overlaps alignment & $x$, $\phi_y$, $\phi_z$ relative to ring & needs beam-halo (Aug--Oct) \\\hline
$\surd$ & Collisions muon alignment & \mbox{6 dof for disks} \mbox{or $x$, $z$, $\phi_y$, $\phi_z$} for chambers & needs collisions ($\ge$ Nov) \\
\end{tabular}
\end{frame}

\begin{frame}
\frametitle{Combining what we have now}

\textcolor{darkblue}{(1) Chamber photogrammetry $\to$ (2) DCOPS/SLM disk bending}
\begin{itemize}
\item Realistic local $x$ positions (assume effect of $\vec{B}$-field is in $z$ and stretching is radial, as symmetry implies)
\item Realistic $z$ bulge and radial bending (at least the $\phi_x$ part, if not $y$)
\item But: no alignment of whole disks in CMS coordinates, which reduces the usefulness of the new local alignment
\end{itemize}

\vspace{0.2 cm}
\hspace{-0.83 cm} \textcolor{darkblue}{\Large Ways to improve this before collisions}

\vspace{0.1 cm}
Any one of the following would improve the alignment; more than one would provide cross-checks:

\begin{itemize}
\item Add a photogrammetry-relative-to-cavern alignment at the beginning \textcolor{darkblue}{(step 1.5)}
\begin{itemize}
\item and again assume that the $\vec{B}$-field effects are $\phi$-symmetric
\item still doesn't connect to the tracker
\end{itemize}
\item Add a transfer line measurement at the end \textcolor{darkblue}{(step 3)}
\begin{itemize}
\item need to propagate tracker coordinate system through the TEC
\end{itemize}
\item Add a standAloneMuon cosmics alignment at the end \textcolor{darkblue}{(step 3 or 4)}
\end{itemize}
\end{frame}

\begin{frame}
\frametitle{Combining steps: implementation}
\begin{itemize}\setlength{\itemsep}{0.35 cm}
\item Every alignment step should do the following:
\begin{enumerate}\setlength{\itemsep}{0.2 cm}
\item read an input geometry in CSCAlignmentRcd format \\ (except chamber photogrammetry because it's first)
\item apply corrections to the parameters the procedure can \mbox{measure;\hspace{-1 cm}} \\ don't modify the parameters that were better determined by previous steps
\item write a modified geometry in CSCAlignmentRcd format
\end{enumerate}

\item If using XML tools, {\tt <moveglobal>}, {\tt <movelocal>}, and
  {\tt <rotatelocal>} provide this functionality

\item This creates a chain of working alignment procedures in which the
  best information is passed through to the end

\item Much easier to understand the results than a global fit that
  includes different methods in a single $\chi^2$ minimization
\end{itemize}
\end{frame}

\begin{frame}
\frametitle{Redundant measurements}
\begin{itemize}\setlength{\itemsep}{0.25 cm}
\item Tracks measure CSC $x$ and $z$ parameters, but not with equal
  precision because they're determined in rather different ways:
\begin{itemize}\setlength{\itemsep}{0.1 cm}
\item $x$ is essentially an average of the whole residuals distribution
\item $z$ is a linear trend in that distribution
\end{itemize}


\item DCOPS measure CSC $x$ and $z$ parameters directly, with $\sim$equal precision because they're measured the same way: \mbox{laser cross-hair\hspace{-1 cm}}

\item Example:
\begin{itemize}\setlength{\itemsep}{0.1 cm}
\item suppose track $x$ accuracy is 0.3~mm and $z$ is 3~mm \\
{\scriptsize (and assume this was determined by comparing different track methods)}
\item suppose DCOPS reproduces $x$ with 1~mm and $z$ with 3~mm
\item in this case, tracks provide 0.3~mm resolution in $x$ \\ and DCOPS provide \underline{1~mm} resolution in $z$
\end{itemize}

\item Cross-checks (e.g.\ DCOPS $x$ above) are not wasted measurements: they
  go into the physics analysis as tighter MC alignment scenarios,
  because we're more confident in the geometry (in $z$ in this case)

\end{itemize}
\end{frame}

\begin{frame}
\frametitle{Available DT procedures}

{\scriptsize (all parameters are local: e.g.\ $x$ is $r\phi$, $y$ is along the beamline, $z$ is radial)}

\renewcommand{\arraystretch}{1.25}

\begin{tabular}{c p{0.25\linewidth} p{0.2\linewidth} p{0.4\linewidth}}
$\surd$ & Photogrammetry/ survey & \mbox{6 dof} \mbox{(stand-alone)} & \mbox{100\% valid at $\vec{B}=0$~T, much} less $\vec{B}$ distortion than endcap \\\hline
$\surd$ & MillePede \mbox{algorithm} & \mbox{6 dof} \mbox{(stand-alone)} & with survey constraint, it's a tracks/survey average: what are the weights? \\\hline
& Hardware \mbox{alignment} & \mbox{6 dof} \mbox{(stand-alone)} & I see that progress is being made \\\hline
$\surd$ & GlobalMuon \mbox{alignment} & \mbox{now: $x$, $y$, $\phi_z$} \mbox{soon: 6 dof} & cosmic rays don't provide complete coverage \\\hline
& StandAloneMuon cosmics alignment & same 6 dof & \mbox{more complete coverage,} \mbox{using aligned chambers as} reference \\
\end{tabular}

\begin{itemize}
\item We've exercised \textcolor{darkblue}{(1) Survey $+$ MillePede $\to$ (2) GlobalMuon} using DTAlignmentRcds for communication
\item I don't yet know which of the 6 parameters will be better determined by which procedure
\end{itemize}
\end{frame}

\begin{frame}
\frametitle{Methods to determine accuracy}
\begin{itemize}
\item When hardware and track-based alignments agree, they're both right \\
(RMS of chamber-by-chamber differences is the sum in quadrature of the accuracies of the two methods)
\item When they disagree, one or both is wrong \\
(pattern of systematic disagreement can be a useful clue)
\item Disagreement or lack of comparison forces us to use internal
  cross-checks to determine accuracy independently
\item Internal cross-checks for track-based alignment:
\begin{itemize}
\item agreement between tracks from different directions and checking consistency of $A-B$, $B-C$, $A-C$ triangles % \mbox{(e.g.\ cosmics)\hspace{-1 cm}}
\item correlation of residuals within chambers, \mbox{discontinuities at borders\hspace{-1 cm}}
\item computing same parameter from different kinds of residuals % \mbox{(e.g.\ $z$ \& $\phi_z$)\hspace{-1 cm}}
\item closure conditions for circular rings of chambers
\item Monte Carlo studies with a given set of systematics
\end{itemize}
\item I expect that hardware and photogrammetry have methods of
  internally determining accuracy of each parameter, too
\item They'll be necessary to objectively decide which to use \mbox{from each\hspace{-1 cm}}
\end{itemize}
\end{frame}


%% \section*{First section}
%% \begin{frame}
%% \begin{center}
%% \Huge \textcolor{blue}{First section}
%% \end{center}
%% \end{frame}

\begin{frame}
\frametitle{Conclusions}
\begin{itemize}\setlength{\itemsep}{0.25 cm}
\item The {\it technical} problem of merging alignment results can be
  solved by each procedure reading and writing in AlignmentRcd format,
  and only modifying best-measured parameters
\begin{itemize}
\item can be done with existing XML tools
\end{itemize}

\item Deciding which alignment procedures have the best measurement of
  a given parameter will require internal cross-checks (determining
  track accuracy using only tracks and determining hardware accuracy
  using only hardware)

\item For the endcaps, we already have enough resolution data to know
  what the combined procedure will probably look like:

\vspace{0.1 cm}
(1) Photogrammetry $\to$ (2) DCOPS disk bending $\to$ (3) Transfer line disk position? $\to$ (4) GlobalMuon alignment in $x$, $\phi_y$, $\phi_z$

\begin{itemize}
\item opportunity to cross-check DCOPS and track-based in $x$ and $z$
\end{itemize}

\end{itemize}
\label{numpages}
\end{frame}

\end{document}
